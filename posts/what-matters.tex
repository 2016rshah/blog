\documentclass{article}

\usepackage[T1]{fontenc}

\usepackage{verse}

\usepackage{epigraph}
\setlength{\epigraphwidth}{.5\textwidth}

\author{Rushi Shah}
\date{26 December 2016}
\title{An Introduction To \\ My Thoughts On Feminism}

\begin{document}

\maketitle

	\epigraph{``At any rate, when a subject is highly contreversial - and any queestion about sex is that - one cannot hope to tell the truth. One can only show how one came to hold whatever opinion one does hold. One can only give one's audience the chance of drawing their own conclusions as they observe the limitations, the prejudices, the idiosyncrasies of the speaker.''}{Virginia Woolf}

	The most bittersweet feeling is when people take my opinion about feminism seriously. On one hand, I feel like I am making a difference by painstakingly convincing a person at a time that sexism is in fact still an insidious and prevalent issue. On the other hand, I worry that they are only taking my opinion seriously because I am a dude. 

	Take, for example, election night 2016: the pent-up breath that went from being an indication of excitement to being released as a sigh and then a sob. That night (and in the coming days, weeks, months) I explained to anyone who would listen why I thought the night didn't just represent a win for Donald, it represented a \textit{loss} for Clinton. 

	%this might be unclear, clarify why Clinton connects to feminism 

	In one particularly memorable moment that night, I was lamenting the role of sexism in the election with a female peer of mine. A male peer of mine listened in on the conversation and explained why he didn't see the influence of sexism on the results. As I explained why I thought he was wrong, and he started nodding, I realized that it just didn't feel right. He had heard it from her, but hadn't concurred until he heard it from me. 

	Somehow we as a society have convinced women that they are not \textit{allowed} to complain about sexism. Although this is a far cry from ideal, the least I can do going forward is utilize the unfortunate authority granted to me by my gender to combat the sexism I see time and time again. 

	% Although this is a far cry from ideal, the least I can do going forward is utilize the unfortunate authority my gender grants me to combat the sexism I see time and time again. 

	% Although this is a far cry from ideal, the least I can until we address that issue is utilize the unfortunate authority granted to me by my gender to combat the sexism I see in the world.

	% Going forward, the least I can do is put this unfortunate comparative advantage I've been given to good use and combat sexism when the opportunity arises.

	% Until we address that issue, the best I can do is call out that which others cannot.

	% But when someone agreed with what I said, it felt problematic. If I was talking to a victim of sexism, what could I possibly tell them about that they haven't or won't personally experience? And if I was talking to a perpetuator of sexism, I felt like I was just reinforcing their predjudices by demonstrating that the only opinions that they should take seriously will come from other men.

	% When faced with the question of being more or less ``radically'' feminist, I've decided to take the 

	% If I'm talking to a victim of sexism, I feel like I'm mansplaining an issue that they should be telling me about (because what could I possibly know that they haven't already personally experienced). If I'm talking to a perpetuator of the issue, I feel like I'm only reinforcing their predjudices by showing that only a man can show them why they are wrong. 

\end{document}
