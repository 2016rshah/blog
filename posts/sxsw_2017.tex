%Preamb stuff that will be used for just about every post:

%Decent margins, layout, and 12 pt font
\documentclass[12pt]{article}
\usepackage[margin=1in]{geometry}

%Syntax highlighting
\usepackage{minted}

%Images directory
\usepackage{graphicx}
\graphicspath{ {../resources/} }

%Colored links
\usepackage[colorlinks]{hyperref}
\hypersetup{
  urlcolor ={cyan},   
  linkcolor={cyan},          % color of internal links (change box color with linkbordercolor)
  citecolor={cyan},        % color of links to bibliography
  filecolor={cyan}      % color of file links
}

\title{South by Southwest 2017: Shifting Focus}
\author{Rushi Shah}
% \date{13 March 2017}

%Post specific stuff

\usepackage{epigraph}
\usepackage{textcomp}

\begin{document}

  \maketitle

      The Turing Scholar Computer Science Honors Program combined with all the math classes I take for my math major gives me plenty of perspective about Computer Science and Math on a daily basis and makes me think about my future in the two fields. But for the past semester that's pretty much all I've thought about. Which is weird, because I really can't imagine not branching out of those two domains at some point in my life. 

      This past week on Spring Break, however, I've been volunteering at South by Southwest, which is a technology/music/film festival that features the who's-who of each industry (along with who's-whos of plenty of other industries). Doing so (attending panels, volunteering, etc.) has put some things in perspective for me. Perhaps this is hubris absorbed from being around so many wildly successful people, but I have become convinced that it is my responsibility to become involved in the public policy process in some way at some point in my life. That's intentionally vague because I don't know exactly what I have in mind (consulting on tech-policy, leveraging tech in public interest, running for office, etc.). There were a combination of a few moments throughout the past few days that have clarified how important this is, though, which I hope to outline in this post. 

      \section{Can Congress Help Tech Diversify - Representative Barragan}

      \section{Norway and the Future of Europe}

        Norway's 5 million people and 432 municipalities: it's all statistics. 

        Aside: we had a really interesting (and bleak) conversation about European geopolitics. Namely, we considered the worst case scenarios: after Brexit and Trump, Marine Le Pen wins in France, Angela Merkel loses in Germany, France leaves the EU, Germany is either left with the mess or the EU disbands, there's a proxy war in Syria between Turkey and Russia, Erdogan tries to recreate the Ottoman Empire by taking over Northern Africa/the middle east, Putin tries to recreate the soviet union by moving further and further into western Europe. UK will probably be fine in the long run, Germany will probably be fine in the long run, the Skandinavian states will band together to form a Skandinavian union post-EU, France might be fine in the long run, but the real losers are countries like Portugal, Greece, and the Balkan states (which are, as usual, stuck between Eastern and Western Europe). 

      \section{Hyperloop and the Physics Major in Congress}

      \section{}

      % CTO for an elected official, consulting on technology policy, or god-forbid running for office myself, at least some part of my life needs to be dedicated to public service.  


\end{document}