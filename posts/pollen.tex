\documentclass[12pt]{article}
\usepackage{cite}

\usepackage[T1]{fontenc}

% \usepackage{minted}

% \usepackage{epigraph}
% \setlength{\epigraphwidth}{0.27\textwidth}

% \newcommand{\resourcepath}{../resources/abstraction-progression-blatex/}

% \usepackage{graphicx}
% \graphicspath{ {\resourcepath} }

\usepackage[colorlinks]{hyperref}
\hypersetup{
  urlcolor = cyan,
  citecolor = cyan
}

\usepackage[margin=1in]{geometry}
% \usepackage{setspace}
% \onehalfspacing

\author{Rushi Shah}
\date{11 June 2016}
\title{Pollination Methods of Orchids and Milkweed}

\begin{document}

\maketitle

I recently attended a ``plant-talk'' at the Botanical Gardens in DC. I don't typically consider myself a biology person, so this was an exciting opportunity for me to branch out a bit. \href{https://www.usbg.gov/events/2016/05/18/lecture-milkweeds-and-orchids-survival-most-efficient}{The lecture} was given by Tatyana Livshultz, Ph.D., Assistant Professor, Drexel University, Assistant Curator, Academy of Natural Sciences and seemed to be loosely based on \textit{The structure and function of orchid pollinaria}~\cite{Johnson2000}. Some of the earlier parts of the lecture went over my head, but the end of the lecture took a look at the implications of the research that was presented, which was very interesting. What follows is a short summary of what I learned. 

\section{Traditional Pollination Method}

  Most flowering plants have dusty pollen monads that cover the insects that land on the plant and brush off onto the subsequent plants the insect visits. There are tons and tons of these little flakes and obviously not all of them reach the plants they were intended for. In fact, only about 1\% of them actually reach their intended destination. This means that there is about a 1\% \textbf{pollen transfer efficiency}  (PTE) for most flowering plants. The rest of the pollen is either lost in transit, or used by insects to feed their offspring. 


\section{Orchids and Milkweeds Pollinate Differently}

  Orchids (Orchidaceae) and milkweeds (Apocynaceae, subfamily Asclepiadoideae) don't pollinate in the same way that typical flowering plants do. Instead of tons of individual pollen monads, they use \textbf{pollinaria}, which are clumps of pollen that are securely attached to insects as they land on the plant. This alternate method of pollination results in a significantly more efficient rate of pollen transfer. Whereas traditional pollination techniques reach about 1\% PTE, pollinaria can reach a rate of 25\% pollen transfer efficiency. 

\section{What accounts for this jump of PTE}

Pollinaria is securely attached to insects as they land on the plant and is delivered directly to the next plant when they land on it as well. (The presentation went into further details about this process, which was slightly too advanced for me to fully grasp). This secure attachment means that the pollinarium does not fall off the insect in flight, and is not even removed by the insect to feed its offspring like typical pollen is. Because the pollen is traveling in a clump, more pollen is also delivered with a successful landing. Because the pollen doesn't fall away like dust and because it isn't used as a food source for insects, orchids and milkweeds can achieve a significantly higher pollen transfer efficiency. 

\section{Why did Orchids and Milkweeds evolve this way}

This higher PTE is important for plants with isolated populations that are only rarely visited. Orchids and milkweeds do not often occur in large colonies of spatially close individuals, they are more isolated from other plants of the same species. Also, insects do interact with them as often and therefore the pollination opportunities are more rare. Thus in response, the evolution of orchids and milkweeds ensured that when they did get the chance to pollinate, they would do so more efficiently. 

\section{Why didn't all plants evolve this way}

Okay great, so using pollinaria for orders of magnitudes better pollen transfer efficiency seems great, all plants should do it, right? Well why don't they? Only the Orchidaceae and Apocynaceae (subfamily Asclepiadoideae) families do so, most other plants pollinate like normal. Turns out that pollinating at a higher efficiency rate with bundled pollen comes at a cost: you're putting all your eggs in one metaphoric basket. You attach your pollen to one insect and you expect it to deliver it $\frac{1}{4}$ of the time, but what if that insect is eaten? Your pollen is lost. Okay it makes it to the next plant, but what if that plant dies? Your pollen is lost. This means that each plant has a lot more riding on one bundle of pollen which can run into issues. In contrast, most plants disperse their pollen much more widely. Even if some methods of pollen transfer fail for them, they will still have large amounts of other pollen monads floating about to carry on. 

%use pollinaria (climps of pollen that they attach to pollinating insects) rather than pollen monads in order to achieve a higher pollen transfer efficiency (PTE). This method is significantly more effective (from a 1\% efficiency for traditional pollen to a 25\% efficiency for pollenaria) which is beneficial for rare and isolated plants that are rarely visited by pollinators. It has this higher transfer rate because the pollinaria is more securely attached to the insect in a way that they cannot remove and does not fall off and because it cannot be used by bees to feed their offspring like traditional pollen is used. However, this higher transfer rate is only found in orchids and milkweeds (as opposed to more species) because it essentially puts all eggs in one metaphoric basket. Having one clump of pollen versus millions of flakes of pollen means that one clump is very important. Perhaps it is lost if the insect is eaten before delivering it. Perhaps it is lost if the flower it is delivered to dies. There is less variation and opportunity. 

\section{???}

Throughout the talk I kept coming back to one question. Given the nature of small grains of pollen they seem rather difficult to track. So how was the pollen transfer efficiency calculated?

% ~\cite{Johnson2000}.

\bibliography{pollen}{}
\bibliographystyle{plain}

\end{document}
