\documentclass{article}
\usepackage{url}
\usepackage{listings}
\title{Against Retribution}
\author{Rushi Shah}
\date{4 May 2021}

\begin{document}

\maketitle

% Intro

    \begin{abstract}

        In this paper, we will first consider the risks of an ideal retributive justice system that portrays punishment as communicative. We will then consider how the risks of such a retributive system rise as it is implemented in a realistic scenario, like the status quo US criminal legal system. Finally, we will consider a far more risk avoidant system of securing justice by drawing on the Black feminist radical tradition of non-carceral ``transformative justice''. Because transformative justice maximizes rank-weighted expected utility in the most risk-avoidant way reasonable, I will argue that it clearly dominates mainstream theories of retributive justice. 

    \end{abstract}

    \section{Introduction}

        In \textit{Risk and Rationality}, Lara Buchak argues on behalf of ``risk weighted expected utility maximization'', comprised of utilities, probabilities, and risk attitudes~\cite{risk-and-rationality}. In \textit{Taking Risks Behind the Veil of Ignorance}, Buchak goes on to ``provide a framework for relating risk and inequality''~\cite{taking-risks-behind-the-veil-of-ignorance} by extending risk weighted expected utility to distributive social gambles during policymaking. She argues that given a choice between two policies X and Y, we should prefer the policy that maximizes weighted-rank utility in the most risk-avoidant way reasonable

        To better understand Buchak's claim, consider a toy example in which Policy X and Policy Y are being compared. Policy Y yields an outcome of utility 4 to everyone and Policy X yields the following utility distribution among the population: 

\begin{lstlisting}[breaklines]
Policy X = {
    1% of population: utility of 7;
    29% of population: utility of 6;
    50% of population: utility of 4;
    20% of population: utility of 1;
} 
\end{lstlisting}

        Notice that U(Policy X) $= (0.01)(7) + (0.29)(6) + (0.5)(4) + (0.2)(1) = 4.01$. Although that is greater than the utility of Policy Y, choosing Policy Y would maximize weighted-rank utility in the most risk-avoidant way reasonable. This is because it would minimize the 20\% risk of an individual experiencing only a utility of 1, and because improving the relatively worse off potential outcomes takes priority over improving the relatively better off ones. 

        In this paper, we will apply this reasoning to select a system of securing justice in response to harm caused by a perpetrator against a victim, such as a criminal justice system\footnote{``Criminal'' is both a politically and philosophically fraught term, so I take care to distinguish crime from harm in this paper.}. Such justice policies will induce risks/benefits for a variety of actors in the system, such as perpetrators, victims, the falsely accused, and potential future victims. Utility, in this sense, represents the benefits and consequences distributed by the policy to such actors: compensation for the victim, burdens of accountability for the perpetrator, and so on. \ We will use utility distributions to make the risks and benefits of a policy more concrete in the following sections. 

        %These risks and benefits will be made more concrete in the form of utility distributions in the following sections. 

        In this paper, we will first consider the risks of an ideal retributive justice system that portrays punishment as communicative~\cite{communicative-retribution}. We will then consider how the risks of such a retributive system rise as it is implemented in a realistic scenario, like the status quo US criminal legal system. Finally, we will consider a far more risk avoidant system of securing justice by drawing on the Black feminist radical tradition of non-carceral ``transformative justice''~\cite{beyond-survival}. Because transformative justice maximizes rank-weighted expected utility in the most risk-avoidant way reasonable, I will argue that it clearly dominates mainstream theories of retributive justice. 


        % TODO: put this somewhere into the body: 
            % 1) false convictions,
            % 2) disproportionate treatment of those who /did/ commit the offense they are accused of.
            % Even a system that made no errors of kind (1) must justify the risks of kind (2) errors it imposes. So when choosing between policies A and B, if B is as effective as A at securing the goods, and equal rates of kind-1 errors,  but A has a higher rate of kind (2) errors, we morally must choose B.

            % "Notice that causing the perpetrator to suffer is not by itself a good secured by the system, it is only seen as a required component of securing other goods on other actors' behalf. Thus, a justice systems also impose the risk of improper treatment even on perpetrators who did in fact commit the harm for which they are accused."

        % Next, we must address the system's response to the perpetrator of harm in this instance. Due, for example, to the imprecision of individual's epistemic states in day-to-day life, it should be clear that the system will at some point be faced with an allegation of harm from a perpetrator to a victim in which the accused individual did not actually perpetrate the harm. We can refer to this as the ineliminable risk of a false accusation, which is a clear downside for the alleged perpetrator\footnote{If not for space constraints, I would also argue there is an additional risk for the perpetrator to be wronged by the system with improper treatment even if they did in fact commit the harm for which they are accused. In other words, even ``deserved'' punishment still wrongs the punished individual. With this additional premise, refraining from punishment reduces the risk of wronging the alleged perpetrator by minimizing the magnitude of harm the system inflicts even on ``non-innocent'' perpetrators, while still securing the necessary goods for the victim.}. This ineliminable risk can be minimized with respect to the probability of the downside (making the mistake less frequent) and with respect to the magnitude of the downside (making the mistake less costly for the wrongly accused).  

        % Thus, this paper will argue that if a system of justice can in fact secure the goods on behalf of the victim while reducing the downside costs imposed on the alleged perpetrator, then that option dominates. First, I will describe how the status quo criminal legal system in the United States of America does not secure the necessary goods on behalf of the victims of crime, and imposes unjustifiable risks on the alleged criminals and Black communities.  Second, I will argue against an idealized hypothetical version of a similarly retributive justice system that portrays punishment as communicative~\cite{communicative-retribution}. Third, I will draw on the Black feminist radical tradition of non-carceral ``transformative justice'' to provide a path forward. Although transformative justice challenges mainstream visions of retributive justice, I will show how it builds community resilience to harm and interrupts rather than escalates the cycle of violence. Because transformative justice secures the necessary goods we expect from a system of justice, and greatly diminishes the risks of grave moral wrong, I will have shown how it clearly dominates mainstream theories of retributive justice. 

    \section{Communicative Theory for Retributive Justice}
        \label{communicative}

        I will now give a breakdown of the risks and benefits of an ideal theoretical retributive justice system. Consider the following utility distribution, in the style of Buchak~\cite{taking-risks-behind-the-veil-of-ignorance}, which represents four actors in a criminal justice system and how their positions change after the execution of retributive punishment against the perpetrator. The specific outcomes secured on behalf of each actor will be explained and analyzed in more depth in this section. 

\begin{lstlisting}[breaklines]
Policy Communicative Retributive Justice = {
    VICTIM: has revenge fantasy satisfied, has relationship with perpetrator completely terminated; 
    PERPETRATOR: suffers from punishment, experiences penance and repentance, experiences social death in community, has no change in criminogenic circumstances; 
    POTENTIAL FUTURE VICTIM: protected from potential perpetrator for length of their incarceration, then reverts to equal or higher chance of being harmed; 
    FALSELY ACCUSED: burdened with trial, suffers from punishment, experiences social death in community;
}
\end{lstlisting}

        On an individual scale, victims can and do want the perpetrators of their harm to suffer. These are valid emotions. As Elisabeth Long describes in chapter 20 of \textit{Beyond Survival}~\cite{beyond-survival}, a ``vent diagram'' can overlay two statements that appear to be true and appear to be contradictory onto a traditional venn diagram. For example, one of Long's vent diagrams labelled the left circle with ``I want the person who raped me to have the community love and support needed to heal, transform, and have the liberated relationships we all deserve'' and labelled the right circle with ``I wish my rapist were dead''. These revenge fantasies, Long argues, have a genuine place in the messy process of change, transformation, healing, and accountability. For this reason, we must count the satisfaction of a revenge fantasy as an outcome secured from the perspective of the perpetrator of harm. 

        However, Long also clarifies that we should not ``equate vengeful feelings with a move toward vengeful action''. Thus, it is still unjustifiable to actually execute personal revenge fantasies. Although we often desire to and do in fact reach for the state monopoly on violence to realize personal revenge fantasies, doing so clearly should not be interpreted as having positive moral utility when evaluating the policy's utility distribution. 

        In \textit{Penal Communications: Recent Work in the Philosophy of Punishment}~\cite{communicative-retribution}, R.A. Duff argues that punishment intended to inflict suffering on the criminal can serve as penance. ``Hard treatment'', he claims, can serve the goals of moral persuasion and reform for the criminal such that they repent. Although I do not find these claims persuasive, he believes that ``kinds of punishment, including those familiar in existing penal systems (community service, fines, probation, even imprisonment), can serve the same communicative purpose, if administered in the right spirit and the right context: they too can force the criminal's attention onto his crime, thus aiming to induce his repentant understanding of what he has done (see Duff 1992). Such punishments can also assist, as well as stimulate, the further process of self-reform and reconciliation.''~\cite[p. 53]{communicative-retribution}. For the sake of argument, we will accept Duff's claims and place a perpetrator's penance and repentance in the outcomes secured by the policy. 

        As described by Lacey and Pickard, modern retributivism places ``an emphasis on the offender's responsibility for blameworthy conduct as core to the permissibility of punishment''~\cite{dual-process}. This emphasis is misplaced, as demonstrated by the way criminogenic circumstances influence the behavior of perpetrators of harm. For example, an individual experiencing homelessness may not have access to bathroom facilities. Their need to urinate will always outweigh the threat of punishment or consequences they may experience. Even if they know they will be subsequently labelled as a sex offender and experience traumatizing incarceration, they will still be physically compelled to urinate. Such criminogenic circumstances need not make the perpetrations of harm justifiable. To provide another example, instances of gender based violence are a consequence of deeply problematic internalizations of intersecting forms of domination, such as heteropatriarchy. Criminogenic circumstances can and do take the form of racism, classism, heteropatriarchy, and imperialism. It is, in some sense, ``rational'' for individuals to remain invested in perpetuating the status quo based on the benefit they derive from their position in such systems of domination. In both cases, retributive punishment makes no progress on addressing the underlying criminogenic circumstances that produced the harm in the first place. Considering punishment for perpetrators of harm in a vacuum and asking how we can persuade them to moral perfection betrays a certain deliberate ignorance towards the circumstances that lead to real instances of harm on a day to day basis. Thus, the failure of the policy to address these root causes is also noted in the outcomes secured on behalf of the perpetrator. 

        % Notice that causing the perpetrator to suffer is not, by itself, a good secured by the system. At best, it is only seen as a required component of securing other goods on other actors' behalf. Thus, justice systems also impose the risk of disutility in the form of improper treatment even on perpetrators who did in fact commit the harm for which they are accused. Go cite Duff where he says that symbolic punishment may be sufficient

        Duff additionally considers the goal of deterrence in a criminal justice system. He correctly identifies that deterrence through the threat of punishment cannot justifiably be an intended good of the system. He says ``if what justifies me in trying to persuade someone to modify her conduct is (my belief) that she ought to do so, the relevant reasons I should offer her are precisely and only those moral reasons that justify my belief that she ought to do so and my attempt to persuade her to do so. If instead I offer her prudential reasons for behaving differently, and particularly if I create those prudential reasons by threatening to inflict harm on her if she remains unpersuaded, I cease to treat or to respect her as a rational moral agent; I am instead trying to manipulate or coerce her into obedience. But is the same not true of a state that seeks to induce its citizens to obey its laws, not by offering them the relevant moral reasons that supposedly justify its laws, but by a system of deterrent punishments that creates new and irrelevant prudential reasons for obedience?''~\cite[p. 14]{communicative-retribution}. Because deterrence cannot justifiably be an intended good of the system, we cannot count it in the slot of benefits secured on behalf of a potential future victim when evaluating the policy's utility distribution\footnote{With that being said, Duff believes deterrence may still be achieved by the communicative retributive justice system insofar as it morally persuades a potential perpetrator away from committing the harm. This is represented in our utility distribution by the penance and repentance of the perpetrator, but still does not permit us to double-count this benefit in the potential future victim slot.}. 

        Retributive punishments, such as incarceration, are often touted to secure the good of incapacitation of the criminal, which will prevent them from reoffending. However, without sentencing the offender to a life sentence of incarceration without the possibility of parole, the offender will eventually be released from prison. The traumatizing effects of prison often put the offender in a more criminogenic position than they were before they entered prison, which contributes to the extraordinary rates of criminal recidivism~\cite{future-of-punishment}. Because the punishment failed to change the criminogenic circumstances of the perpetrator, the preventative good of punishment is no longer secured for the potential future victim after the perpetrator is released. 

        Given the scale on which such a system would be implemented, it will undoubtedly face cases in which the accused individual did not perpetrate the harm. Additionally, the system will fail to recognize the accusation as false in an ineliminable number of cases. A policy can reduce the risks to the falsely accused not only by reducing the frequency of such mistakes, but also by minimizing the magnitude of harm done by such a mistake. Without empirical evidence, it is hard to make claims about the frequency of false accusations across policy options. However, retributive punishment implements ``hard treatment'', which makes the risk of grave suffering by the falsely accused unacceptably high. 

        At this point in the argument, we have seen through a utility distribution how a reasonable version of theoretical retributivism imposes serious costs and risks, especially relative to the goods it achieves for victims. However, the costs of a retributive justice only seem to rise when implemented in a real world setting. In the next section, we will amend the retributive utility distribution to acknowledge a fifth actor for whom outcomes are secured in the American criminal legal system: the punishment profiteer. 


    \section{The American Criminal Legal System}
        \label{criminal-legal-system}

        Although this paper is largely concerned with theoretical conceptions of justice in the ideal sense, I would be remiss to ignore the realities of the retributive system with which we intend to administer justice in the United States of America. The ways the US criminal legal system treats victims, perpetrators, potential future victims, and the falsely accused can be modeled with the theoretically ideal system of retributive justice presented in Section~\ref{communicative}. However, as it is currently implemented, we must consider an additional actor to represent the portion of the population that undeservedly benefits from the institution of punishment itself, independent of any particular instance of harm. This actor, which represents the prison industrial complex, extracts benefits from the carceral state at the expense of the other actors in the system. We will refer to this actor as the punishment profiteer because they benefit from the infliction of punishment, and thus are incentivized to promote as much and as drastic punishment as possible within the retributive system. The realistic risks imposed by such punishment profiteers, and the lack of countervailing weighted rank utility upside, strengthens the argument against retributive justice more broadly. 

\begin{lstlisting}[breaklines]
Policy US Criminal Legal System = {
    VICTIM: has revenge fantasy satisfied, has relationship with perpetrator completely terminated; 
    PERPETRATOR: suffers from punishment, experiences penance and repentance, experiences social death in community, has no change in criminogenic circumstances; 
    POTENTIAL FUTURE VICTIM: protected from potential perpetrator for length of their incarceration, then reverts to equal or higher chance of being harmed; 
    FALSELY ACCUSED: burdened with trial, suffers from punishment, experiences social death in community;
    PUNISHMENT PROFITEER: solidifies political power, exploits larger supply of subordinated labor, employs rural workforce in prisons, financially loots perpetrator and their community,  bolsters white supremacy with ritualized antiblack violence;
}
\end{lstlisting} 


        % On a larger political scale, it is convenient to scapegoat the problems of a broken society onto those most representative of its flaws. For example, wealthy restaurant-goers may find the existence of individuals experiencing homelessness close to their dining experience distasteful. In the status quo, such individuals are punished for their poverty by being arrested for living their private lives in public spaces, such as urinating in public when they are blocked from accessing proper facilities. The restaurant-goers do not feel as strongly about the problem of homelessness being solved as they do about the problem of homelessness simply not being visible to them. The carceral state is in fact able to secure this ``good'' for the restaurant-goers by punishing the homeless individual for existing too close to the restaurant-goers, even if we think the good has limited or negative philosophical and moral value. To challenge the status quo, then, we can not simply argue against its effectiveness at achieving its goals, we must argue against the goals themselves. 

        % This contrast between effectiveness-critiques and goals-critiques is even more evident when one considers the variety of roles the carceral state plays as a tool for white supremacy. This includes the preservation of white electoral power, racial capitalism for the prison industrial complex, and the afropessimistic lens of ritualized political violence. 

        To see how punishment plays a role in the preservation of electoral political power, consider how states within the US often disenfranchise individuals convicted of felonies. To demonstrate the point, we will use Texas as a representative example~\cite{daily-texan}. Whites, Blacks and Latinos each make up roughly a third of the Texas incarcerated population, which totals about 250,000 people, despite Blacks only representing 12\% of the overall Texas population~\cite{prison-policy-tx}. After disproportionately locking up Black and Latino individuals, prison facilities grant the voting power of the disenfranchised to the communities within which the prisons are built. This is because the state will revoke the voting rights of the incarcerated individual but still count them as part of the community when determining how much representation an area receives in elections. The Houston Chronicle reports that 70\% of prisons during the United States prison boom were built in rural communities~\cite{houston-chronicle}. This geographic setup allows white rural communities in Texas, the punishment profiteers, to skew electoral representation in their favor by absorbing the voting power of incarcerated communities of color. 

        Similarly, punishment plays a central role in racial capitalism for the prison industrial complex. Angela Davis argues in chapter 5 of \textit{Are Prisons Obsolete} that both public and private prisons provide a subordinate supply of labor: ``For private business prison labor is like a pot of gold. No strikes. No union organizing. No health benefits, unemployment insurance, or workers’ compensation to pay. No language barriers, as in foreign countries. [...] Prisoners do data entry for Chevron, make telephone reservations for TWA, raise hogs, shovel manure, and make circuit boards, limousines, waterbeds, and lingerie for Victoria’s Secret, all at a fraction of the cost of `free labor' ''~\cite{are-prisons-obsolete}. In \textit{The Golden Gulag}, Ruth Wilson Gilmore documents carceral geographies and how prisons are often pitched as solutions to economic downturns in rural communities~\cite{golden-gulag}. Finally, Jackie Wang argues in \textit{Carceral Capitalism} that the American criminal justice system uses punishment as a tool to extract and loot primarily Black communities. She says ``While extraction and looting are the lifeblood of global capitalism, it occurs domestically in the public sphere when government bodies--out of pressure to satisfy their private creditors--harm the public not only by gutting social services, but also by looting the public through regressive taxation, fee and fine farming, offender-funded criminal justice ``services'' such as private probation services, and so forth. While in the private sector the extension of subprime credit is often deployed as a racialized form of expropriation, in the public sector municipal governments (in tandem with or on behalf of financial institutions) use the police and the criminal justice system to loot residents of primarily black jurisdictions''~\cite[p.76]{carceral-capitalism}. In these cases, the prison industrial complex corporations, the rural communities, and the public sector municipal governments are the punishment profiteers who impose the risk of costs on other members of the population for their own disproportionate benefit. 

        Punishment also functions as a tool of ritualized political violence. Using the afro-pessimistic lens to analyze ritualized political violence, we can see gratuitous violence as a defining and central feature of antiblack racism. This gratuitous violence can take the form of lynchings during Jim Crow, contemporary videotaped police shootings, and inconsistently successful mass movements to stay state executions of individuals on death row, all for alleged violations of criminal laws. Wang argues on behalf of afro pessimists that ``whiteness as a category is, in part, maintained by ritualized violence against black people and white consumption of spectacularized images of antiblack violence''~\cite[p.92]{carceral-capitalism}. Punishment, both judicial and extra-judicial, then functions to perpetuate the existing forms of racial domination and subordination~\cite{crt}. Punishment profiteers, in this case, derive their ``profit'' in the form of further insulation for the dominant racial hierarchy of white supremacy. 

        With this additional actor in the utility distribution, the weighted rank utility argument comes more clearly into focus. First of all, it should be clear that the punishment profiteers extract their profit at the expense of the other actors in the system, even if those expenses are not explicitly noted in the utility distribution. For example, the victims of gratuitous violence may be largely selected from the pool of perpetrators and potential perpetrators who are all vulnerable as a result of their designation in the retributive system as criminal~\cite{criminalization-of-blackness}. This makes the magnitude of the ``suffers from punishment'' outcome in the utility distribution substantially worse for the punished than it may have been conceptualized as in Section~\ref{communicative}. Second of all, it should be clear that having actors clearly incentivized to maximize punishment, as the punishment profiteers are, substantially increases the probability of the risks associated with unjustifiable punishment. It is in the best interest of the punishment profiteers to make the perpetrator/falsely accused class as large as possible, which increases the probability of the risks they bear. Third of all, it should be clear that any weighted ranking should deprioritize the further enrichment of the punishment profiteers. In the next section, we will demonstrate a policy option that does not take punishment as an a priori operating principle, and leads to a clearly preferable weighted rank utility when compared to the utility distributions of the retributive systems we've seen so far.  

        % In these ways, the philosophical merit of retributive justice is a red herring with respect to the other, philosophically less savory but politically more pertinent, goals of American punishment. That is not to say that punishment is ineffective or pointless or improperly implemented. Rather, that is to say that punishment in our existing criminal legal system is successfully functioning to achieve some tangible goods of dubious value~\cite{crt}, even if those goods do not serve the victim of harm. 

        % retribution as an outlet for anger/frustration (just deserts)
            % on an individual scale, victims of harm can and do want their perpetrators to suffer. These are valid emotions. 
                % But they do not justify state power to realize their personal revenge fantasies. These emotions do not endow the state with the standing to punish
                    % unconvinced, for obvious reasons, that the state must do it to avoid vigilanteism. 
                % The state's job is to smooth out the emotional variance of individuals and optimize for community benefit.
            % on a larger political scale, it is convenient to scapegoat the problems of a broken society onto those most representative of it's flaws
                % This political instinct misunderstands the source of the problems. Perpetrators of harm are often motivated by the perverse incentives of their society. To focus on the individual is to ignore the source of the problem, even if it is politically popular
                % NIMBY's on the problem of homelessness
        % retribution as a tool for white supremacy
            % preservation of political power
                % felon disenfranchisement for voting power
            % the prison industrial complex
                % racial capitalism
                % private AND PUBLIC prison as a source of labor (are prisons obsolete)
                % carceral geographies 
            % ritualized political violence
                % afropessimism
                % "whiteness as a category is, in part, maintained by ritualized violence against black people and white consumption of spectacularized images of antiblack violence" (Carceral Capitalism 92)
                    % lynchings
                    % police violence
            % critical race theory and critical legal studies
                % supposedly impartial legal doctrines are actually tools to insulate structures of subordination/domination. 

            % policing to maintain the heteropatriarchal racial capitalist hierarchy
            % afro pessimism and critical race theory
            % race for profit
        % traumatizing effects of incarceration? retraumatizing effects on the victim
        % not successfully serving the needs of the victim 



    \section{Transformative Justice (TJ)}
        \label{transformative-justice}

        The current criminal legal system in America poses disproportionate and substantial risk of harm to certain communities, such as undocumented individuals, the Black community, and sex workers, to name a few. These communities have organized to address the harms they experience without the option to mobilize a state response~\cite{tj-brief-description}. For example, Black women pioneered ways to address gender based violence in their community without calling the cops, because carceral feminism~\cite{carceral-feminism} was not in their or their communities' best interest. Such non-carceral approaches to responding to harm offer a path towards accountability without the underlying assumption that punishment is justified for perpetrators of harm~\cite[p.203]{beyond-survival}, and collectively comprise ``transformative justice'' (TJ). 

        Compared to the substantial costs and risks of retributive justice systems, both realistic and ideal, transformative justice offers an alternative system that achieves the relevant goods while imposing fewer costs and risks. In describing transformative justice, I will once again start with a high level utility distribution with the same actors from before to answer the following question: how does the policy change the marginal position of each actor after an instance of harm is addressed with the policy?

\begin{lstlisting}[breaklines]
Policy Transformative Justice = {
    VICTIM: emotionally and materially heals, persuaded away from revenge fantasy, rebuilds relationship with perpetrator if desired; 
    PERPETRATOR: burdened with requirements of accountability, heals from criminogenic circumstances (rehabilitation of self and of situation), reconciled with community; 
    POTENTIAL FUTURE VICTIM: protected from future harm;
    FALSELY ACCUSED: burdened by deliberative process, burdened non-punitively by false accountability; 
    PUNISHMENT PROFITEER: no change in political power, no change in subordinated labor supply, no additional rural prison jobs, no financial gain from perpetrator and their community, no opportunity for strengthening white supremacy with ritualized antiblack violence;
}
\end{lstlisting}

        Transformative justice defies definition\footnote{To make it slightly more concrete, you can, for the time being, imagine transformative justice as a deliberative dialogue within the community about what harm occurred, why it occurred, and what must be done going forward.}, but has some components that seem essential to its character. Ultimately, transformative justice provides a framework to respond to instances of harm in a community. In doing so, transformative justice centers the needs of the victim of harm, interrupts rather than escalates the cycle of violence, addresses the root causes of the harm that took place, and rejects carceralism and retribution. 

        Transformative justice centers the needs of the victim of harm. The need to conceptually separate the notion of justice for the victim from punishment for the perpetrator has even been supported by retributivists, like the advocates for a dual-process approach to criminal law~\cite{dual-process}. Early in the transformative justice process, the victim can set goals they would like the process to achieve, which can help the facilitators tailor each process to the specific harm and individual. These goals can relate to how the victim needs to process the trauma they experienced, what services they need access to support them as they heal, what support they need from their community, and what they want for the person who harmed them. The goal setting process can also be done as a group in the transformative justice process, which can give space to allies and even potentially the perpetrator of harm to name what they would like to achieve in the transformative justice process. For each goal, the goal-setter can ask themselves whether or not the goal fits their values, whether or not it will lead to more harm for themselves or others, and how achievable it realistically is.\footnote{More details on goal setting can be found in chapter nine of \textit{Beyond Survival}, such as sample goals and guided questions~\cite{beyond-survival}.} After the goals are set, the process can proceed in a way that respects the goals of the victim in that specific scenario, rather than applying a one-size-fits-all response to harm. This process helps the victim heal from the harm both materially and emotionally, persuades them away from indulging their revenge fantasies, and allows them to maintain a relationship with the individual who has harmed them if that is what the victim wants. 

        Transformative justice interrupts rather than escalates the cycle of violence. In the sense that harm is violence done by the perpetrator against the victim, transformative justice does not respond to harm with retributive violence, by the state or otherwise, against the perpetrator. This component of transformative justice comes from a rejection of the false victim/perpetrator binary. Perpetrators of harm have, in other instances, been the victims of harm themselves. Similarly, victims of harm have been and will be perpetrators of harm in other instances as well.\footnote{The rejection of this false dichotomy, however, does not mean transformative justice discounts the circumstances of a particular instance of harm, blames the victim for what happened, or ignores the role the perpetrator played.} These two facts play a role in an escalating cycle of violence in which harm in one instance leads to worse harm in the next. Responding to harm with the state monopoly on violence in the form of punishment does the opposite of interrupting this cycle. In contrast, transformative justice aims to serve in the best interest of every individual in the entire community (including the perpetrator of harm), because every individual can play every role in the system at some point. By treating the perpetrator of harm with this level of grace and love, rather than with wrath and scorn, the system is able to be a stopping point in the otherwise escalating cycle of violence.\footnote{On grace, love, wrath, hatred, see chapter three of \textit{Beyond Survival}~\cite{beyond-survival}. It is titled ``Isolation Cannot Heal Isolation'' and in it, Blyth Barnow writes the following of her abuser: ``I wrote the letters that night, and as I did, I reconciled a few things. One, I would always believe he was worth it. Two, I deserved as much support as I wanted to give to him. And three, it was not my job to take the lead in his healing.''} This rejection means that the perpetrator will be faced with the minimal sufficient burdens of accountability: their risk of suffering at an unacceptable magnitude goes down as the cycle of violence is broken.

            % TODO: talk about the falsely accused at the end of the previous paragraph (maybe in a footnote) 

        Instead of executing retribution, transformative justice aims to address the root causes of harm. The reasons why perpetrators commit harm can be complex and nuanced, but it is a key premise of transformative justice that perpetrators of harm do in fact have reasons for doing what they do. These root causes are critical to understanding why harm happens, and how to prevent it in the future. Often, the existential fear of retribution prevents perpetrators from reckoning with their role in harming others. To address this obstacle, the role of perpetrators of harm in the process must shift away from them being targets of punishment, and towards them being community members similarly collectively engaged in transforming culture and circumstances. This shift allows perpetrators to genuinely reconcile with their community. Perpetrators can then accept the responsibility they deserve while also naming what community support they need to prevent future harm. In this way, the perpetrator must deal with their burdens of accountability, but the society also must change in a way that addresses the perpetrator's criminogenic circumstances. After both of these requirements have been satisfied, the risk to the potential future victim are significantly reduced because there are no longer extant causes for them to be harmed. 

        Transformative justice rejects retributive punishment, such as carceralism. This helps suppress the unfortunate power-based and racial realities of the status quo criminal legal system described in Section~\ref{criminal-legal-system}. Clearly, the risk of wrongful punishment disappears if the system does not punish anyone. Similarly, the risk of disproportionate punishment for actual perpetrators also disappears. And, of course, the punishment profiteers can no longer derive their profit at the expense of the other actors in the system, which diminishes all associated risks. 


            % "a non-carceral approach for responding to harm"
            % victim centering
            % interrupting the cycle of violence
                % rejecting the victim/perpetrator dichotomy
                    % victims of harm in one instance are perpetrators of harm in another
                    % perpetrators of harm in one instance have been victims of harm in the past
                    % this does not erase the tangible instance of harm, but it motivates why justice must have the best interests of both parties (and the broader community) in mind
                % invoking state violence only escalates the magnitude of violence involved
            % healing for all members of the community (including the perpetrator)
            % no punishment
                % fear of retribution prevents accountability, so removing it creates space for perpetrators to accept the responsibility they deserve and also name what community support they need to prevent perpetrating future harm
            % deliberative dialogue that centers the victim and the harm they experienced, and can include the perpetrator 

    \section{Conclusion}

        % tie it back to risk and justifiability with risk weighted expected utility maximization

        Transformative justice, as a rejection of retributive justice, should be seen as an opportunity to explore the depths of rehabilitation for the community as a whole (victim and perpetrator of harm included). It builds community resilience to harm by providing the support victims need to heal, treats the perpetrator as a community member equally invested in transforming the community going forward, and builds the capacity to prevent future harm caused by the same root problems and criminogenic circumstances. In this way, transformative justice is able to secure goods purportedly secured by theoretical and realistic systems of retributive justice, with far fewer risks. For this reason, when you compare the utility distributions in Sections~\ref{communicative}, \ref{criminal-legal-system}, and \ref{transformative-justice}, it is clear that transformative justice should be preferred to maximize risk weighted utility. 

        % opportunity to explore the depths of rehabilititation
        % builds community resilience to harm and capacity to prevent future harm
        % risk of wrongful punishment goes away if you don't punish anyone
        % address criminogenic circumstances

        Such a radical revision to standard theories of justice may seem dauntingly out of reach. However, transformative justice operates at the granularity of communities, which can be conceptualized as concentric circles spreading outwards from an individual. The primary components of transformative justice I outlined in Section~\ref{transformative-justice} can be incrementally practiced by any such individual in their own communities and organizations. Although we instinctively want to scale such systems to the largest possible population, transformative justice can provide immediate benefits from the mental shift away from a notion of justice saturated with retribution. 

        As mentioned at the beginning of Section~\ref{transformative-justice}, transformative justice has been being practiced by Black women and other communities who do not enjoy the benevolence of the state. We can take these stories as inspiration for the practice of our own lives. For example, consider Patrice Cullors' experiences described in \textit{Abolition and reparations: Histories of resistance, transformative justice, and accountability}: ``I have sat in, facilitated, and participated in many healing circles with people I’ve harmed and who have harmed me. Defensiveness, anger, self-righteousness, self-realization, serenity, and other emotions have come over me and through me in those moments. I am grateful for the opportunities I have had to apologize and learn from my mistakes. I am appreciative of the times I have forgiven and moved beyond the harm, toward transformation.''\cite{histories-of-tj}

        % radical revision of standard theories of justice
        % incrementalism and the deprioritization of scale 
        %   cite the HLR article about practicing abolition https://harvardlawreview.org/2019/04/abolition-and-reparations-histories-of-resistance-transformative-justice-and-accountability/

\bibliography{ref}
\bibliographystyle{plain}
\end{document}