\documentclass[12pt]{article}

\usepackage{listings}
\lstset{
  language=Haskell
}

\usepackage{graphicx}
\graphicspath{ {../../resources/} }

\usepackage[margin=1in]{geometry}

\usepackage{hyperref}

\newcommand{\tightlist}{\setlength{\itemsep}{0pt}\setlength{\parskip}{0pt}}

\title{Clearing Up Week 05 of CIS194}
\author{Rushi Shah}
\date{8-30-2015}

\begin{document}

  \maketitle

So \href{http://www.seas.upenn.edu/~cis194/spring13/lectures/05-type-classes.html}{week
five of CIS194} was a roller coaster! It really gets down with the
nitty-gritty of Haskell's type system, which is great! But it can also
be a bit confusing. I spent a really long time on this week because I
know how important it is. With that being said, a lot of time is spent
in the
\href{http://www.seas.upenn.edu/~cis194/spring13/hw/05-type-classes.pdf}{assignment
pdf}, but there isn't THAT much material in the lecture. So here is a
bit of clarification on everything mostly just for my own edification.
Any benefit you derive is just a side-effect (hah, puns!).

\subsection{Why I was confused}\label{why-i-was-confused}

In my mind, there were four things and four pieces of syntax.

The four things:

\begin{itemize}
\tightlist
\item
  \textbf{Types}
\item
  \textbf{Type classes}
\item
  Types are \textbf{instances} of type classes
\item
  \textbf{Type synonyms} are types that are exactly the same thing as
  other types
\end{itemize}

The four pieces of syntax (with examples from the lecture/assignment of
CIS194 week 05):

\begin{itemize}
\tightlist
\item
  \texttt{data\ Foo\ =\ F\ Int\ |\ G\ Char}
\item
  \texttt{instance\ Eq\ Foo\ where}
\item
  \texttt{class\ Eq\ a\ where}
\item
  \texttt{type\ MyChar\ =\ Char}
  \begin{itemize}
    \tightlist
    \item
      The last one was not actually really a part of week 5, but it exists and was
      floating around in my head, so I have included it.
  \end{itemize}
\end{itemize}

\subsection{How I cleared everything
up}\label{how-i-cleared-everything-up}

This is how things match up with their syntax (along with an explanation
of the order things generally are used):

\subsubsection{Type and Data}

\begin{itemize}
\tightlist
\item
  Define a type for the thing (\texttt{MyMaybeInteger})
  \begin{itemize}
    \item
      Use the \texttt{data} syntax
  \end{itemize}
\end{itemize}

\begin{lstlisting} 
data MyMaybeInteger = J Integer | N
--J for Just 
--N for Nothing 
\end{lstlisting}

\begin{itemize}
\tightlist
\item
  Define a type class for what attributes the thing should have
  (\texttt{MyEq})
  \begin{itemize}
    \item
      Use the \texttt{class} syntax
  \end{itemize}
\end{itemize}

\begin{lstlisting} 
class MyEq a where 
  equal :: a -> a -> Bool 
  notEqual :: a -> a -> Bool 
\end{lstlisting}

\begin{itemize}
\tightlist
\item
  After you have both of those, constrain the type to the type class
  \begin{itemize}
    \item
      Use the \texttt{instance} syntax
  \end{itemize}
\end{itemize}

\begin{lstlisting} 
--Read as ``an instance of MyEq is MyMaybeInteger'' 
instance MyEq MyMaybeInteger where 
  equal N N = True
  equal (J a) (J b) = a == b 
  equal _ _ = False 
  notEqual a b = not (equal a b) 
\end{lstlisting}

\begin{itemize}
\tightlist
\item
  Side note: if you want a type synonym, use the \texttt{type} syntax.
\end{itemize}

I guess what really got me mixed up was that to define a new type, you
use the \texttt{data} syntax, and the \texttt{type} syntax is just for
type synonyms.

\begin{lstlisting} 
type MyInteger = Integer 
\end{lstlisting}

\subsection{Conclusion}\label{conclusion}

So there you have it, a bit more clarification on Week 05 of CIS194. It
will definitely take me a while to wrap my head around Haskell's type
system, but I am excited because it seems very mature and I can't wait
to discover its intricacies. To keep track of my efforts on CIS194,
check out the \href{https://github.com/2016rshah/CIS194}{github repo}
and \href{https://github.com/2016rshah/CIS194/tree/master/05}{week 05
specifically}.

\subsection{\texorpdfstring{Post Note:
\texttt{newtype}}{Post Note: newtype}}\label{post-note-newtype}

It's been a while since I originally wrote this post, but it has come to
my attention that there is actually one more thing (not explicitly
mentioned in Week 05) that needs to be mentioned. The
\href{https://wiki.haskell.org/Newtype}{\textbf{newtype}} syntax is very
similar to the data syntax.

\begin{quote}
The syntax and usage of newtypes is virtually identical to that of data
declarations - in fact, you can replace the newtype keyword with data
and it'll still compile, indeed there's even a good chance your program
will still work. The converse is not true, however - data can only be
replaced with newtype if the type has exactly one constructor with
exactly one field inside it.
\end{quote}

Basically ``if you want to declare different type class instances for a
particular type, or want to make a type abstract, you can wrap it in a
newtype''. A good example is using
\texttt{newtype\ Email\ =\ Email\ String}. The idea behind that is that
your code is more expressive and it adds a layer of typesafety (if a
function is expecting an Email, you can't pass it any old string).

But according to this
\href{http://degoes.net/articles/newtypes-suck/}{eloquent rant on what's
wrong with newtypes}, that is only a false sense of security. The gist
of the article is that doing something like
\texttt{newtype\ Email\ =\ Email\ String} only makes you think your code
is more safe because anybody (including you) can wrap a string that is
clearly not an email into the newtype and assume it is one which brings
you back to square one. The moral of the story is that although newtypes
might help, and are useful, they aren't foolproof.

\end{document}