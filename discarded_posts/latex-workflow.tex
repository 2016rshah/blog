\documentclass[12pt]{article}

\usepackage{listings}
\lstset{
  language=Bash
}

\usepackage{graphicx}
\graphicspath{ {../resources/latex-workflow/} }

\usepackage[margin=1in]{geometry}

\usepackage[normalem]{ulem}

\usepackage[colorlinks]{hyperref}
\hypersetup{
  urlcolor = {cyan}
}

\newcommand{\dollar}{\$}

\title{A Beginner's Haskell Workflow in Emacs}
\author{Rushi Shah}
\date{5 December 2015}

\begin{document}

  \maketitle

% \begin{figure}[htbp]
% \centering
% \includegraphics{http://imgs.xkcd.com/comics/file_extensions.png}
% \caption{XKCD}
% \end{figure}
\begin{center}
\includegraphics[height=2.5in]{file_extensions}
\end{center}

When it came time to write college essays, I decided to write them all
in LaTeX. Is LaTeX a bit overkill for something that normal people would
use Microsoft Word for? Yeah probably. But

\begin{itemize}
\item
  I wanted to learn LaTeX anyways
\item
  My editor configuration is so much better than the un-customizable
  Microsoft Word
\item
  I can easily version-control my essays with git since \texttt{.tex}
  files aren't rich-text like \texttt{.docx} files
\end{itemize}

So I continued on my merry way and happily wrote my essays in LaTeX,
created a git repository, used
\href{https://github.com/SublimeText/LaTeXTools}{Sublime Text's
LaTeXTools} for the build system, etc. But the issue arose when I tried
to send my essays to my family to look over. My sister wasn't a fan of
the pdf format I had sent her, but she humored me. My parents, on the
other hand, refused to look at anything other than Microsoft Word files.
If I wanted to continue to use LaTex, I had to find a way to convert the
files to word before sending them to my family.

\section{Pandoc}\label{pandoc}

The obvious answer was \href{http://pandoc.org/}{Pandoc}. It is terrific
and easily handled the task of converting \texttt{.tex} to
\texttt{.docx}. And best of all, it is written in Haskell! But it is
still a command line utility, so I needed a way to set it up to
automatically make the conversion when needed. I could have created a
\href{http://gulpjs.com/}{gulp task} that watched the files for changes
and made the conversion. But there was no point in converting every time
I saved because I would work on the file for a while before I was ready
to send it to my family. At most, I would need a converted word file
every time I made a change significant enough to commit to git
(\href{http://www.databasically.com/2011/03/14/git-commit-early-commit-often/}{commit
early and often, kids!}). So ideally, the converting would be bundled
with the action of adding a file, commiting it, and pushing to my Github
remote (which is unfortunately still a private repository).

\section{The Bash Script}\label{the-bash-script}\indent

The most elegant solution I could think of was to just create a bash
alias for that action.

The first step was a wrapper function that would take as many files as
needed and convert them all to word format, saving the resulting file in
a folder that can be added to the \texttt{.gitignore}. The feature of
converting an arbitrary number of files at a time is unnecessary, but
doesn't hurt.

\begin{lstlisting}
latToDoc(){
    #Usage: `latToDoc FILE_NAME`                                                                                                        
    mkdir -p word_format #In case the folder doesn't exist yet                                                                          
    for arg in "\dollar@"
    do
        echo "Converting \dollararg"
        pandoc -f latex -t docx \dollararg.tex -o word_format/\dollararg.docx
        open word_format
        echo "Converted \dollararg"
    done
}
\end{lstlisting}

Then, I needed a quick function that would convert the file, add it,
commit it with the provided commit message, and push it to the remote.

\begin{lstlisting}
pushAndConvert(){                                                                                                                       
    #Usage: `pushAndConvert FILE_NAME "COMMIT_MESSAGE"`                                                                                 

    latToDoc \dollar1                                                                                                                         

    git add \dollar1.tex \dollar1.pdf                                                                                                               
    echo "Added \dollar1.tex and \dollar1.pdf"                                                                                                      

    git commit -m "\dollar2"                                                                                                                  
    echo "Committed with message: \dollar2"                                                                                                   

    git push origin master                                                                                                              
    echo "Pushed to master"                                                                                                             
}
\end{lstlisting}

Now all I do is
\texttt{pushAndConvert\ accept\_me\_please\ "Doubled\ bribery\ offer"}.
Simple enough, right?

\end{document}